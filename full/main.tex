\documentclass{article}
\usepackage{amssymb}
\usepackage{amsmath}
\usepackage{array}
\usepackage[margin=2.0cm]{geometry}
\usepackage{xeCJK}
\setCJKmainfont{AR PL UKai TW}
\title{筆記}
\author{陳定善}
\date{}
\newcommand{\sd}[1]{{\left(#1\right)}}
\newcommand{\df}{定義:}
\newcommand{\thm}{定理:}
\newcommand{\axm}{公理:}
\newcommand{\anno}{註:}
\newcommand{\md}[1]{{\left[#1\right]}}
\newcommand{\prer}{前提:}
\newcommand{\prop}{性質:}
\newcommand{\pf}{證明:}
\newcommand{\abs}[1]{{\left|#1\right|}}
\begin{document}
\maketitle
\section{logic 邏輯}
這一章只會簡單介紹會用到的邏輯符號及基本的一些公理(axiom)與定理(theorem)。\\
通常上,「對」會表示為$T$或是$\top$,而「錯」會表達為$F$或是$\bot$。在這篇文章當中,會以$\top$及$\bot$表達。\\
首先,「公理」是對該話題的預先假設,而「定理」是從假設中推論出來的,而定理會附帶證明。通常,推論會寫成$B,C\vdash D$,意味著以$B, C$為前提推論出$D$。\\
至於推論與假設代表什麼,我還不知道該如何解釋。\\
\axm
\begin{equation}\label{logic:A1}
	B \to \sd{C \to B}
\end{equation}
\axm
\begin{equation}\label{logic:A2}
	\sd{B \to \sd{C \to D}} \to \sd{\sd{B \to C} \to \sd{B \to D}}
\end{equation}
\axm
\begin{equation}\label{logic:A3}
	\sd{\sd{\lnot B} \to \sd{\lnot C}} \to \sd{\sd{\sd{\lnot B}\to C}\to B}
\end{equation}
\axm
\begin{equation}\label{logic:MP}\tag{MP}
	\sd{B \to C}, B \vdash C
\end{equation}
以上四個公理中,$B,C,D$是任意敘述。以上公理,可以理解成是在對$\to,\lnot,\vdash$做定義,只要符合以上公理形式的概念,都是可以使用的。\\
\thm
\begin{equation}\label{logic:to_self}
	B \to B
\end{equation}
\pf\\
\begin{tabular}{ >{\raggedright\arraybackslash}p{0.1\textwidth}  >{\centering\arraybackslash}p{0.7\textwidth} >{\raggedleft\arraybackslash}p{0.2\textwidth}}
$C_{1}:$&$\sd{B \to \sd{\sd{B \to B}\to B}} \to \sd{\sd{B \to \sd{B \to B}} \to \sd{B \to B}}$&$(\ref{logic:A2})$\\
$C_{2}:$&$B \to \sd{\sd{B \to B} \to B}$&$(\ref{logic:A1})$\\
$C_{3}:$&$B \to \sd{B \to B}$&$(\ref{logic:A1})$\\
$C_{4}:$&$\sd{B \to \sd{B \to B}} \to \sd{B \to B}$&$(C_{1}, C_{2}, \ref{logic:MP})$\\
$C_{5}:$&$B \to B$&$(C_{3},C_{4},\ref{logic:MP})$\\
\end{tabular}
\thm

\end{document}
