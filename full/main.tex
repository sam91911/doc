\documentclass{article}
\usepackage{amssymb}
\usepackage{amsmath}
\usepackage{array}
\usepackage[margin=2.0cm]{geometry}
\usepackage{xeCJK}
\setCJKmainfont{AR PL UKai TW}
\setlength{\tabcolsep}{0pt}
\title{筆記}
\author{陳定善}
\date{}
\newcommand{\sd}[1]{{\left(#1\right)}}
\newcommand{\df}{定義:}
\newcommand{\thm}{定理:}
\newcommand{\axm}{公理:}
\newcommand{\anno}{註:}
\newcommand{\md}[1]{{\left[#1\right]}}
\newcommand{\prer}{前提:}
\newcommand{\prop}{性質:}
\newcommand{\pf}{證明:}
\newcommand{\abs}[1]{{\left|#1\right|}}
\newcommand{\rom}[1]{\romannumeral #1 \relax}
\newcommand{\pftab}[1]{
	\begin{tabular}{ >{\raggedright\arraybackslash}p{0.15\textwidth}<{}  >{\centering\arraybackslash}p{0.7\textwidth} >{\raggedleft\arraybackslash(}p{0.15\textwidth}<{)}}
		#1
	\end{tabular}\\
}
\begin{document}
\maketitle
\section{logic 邏輯}
這一章只會簡單介紹會用到的邏輯符號及基本的一些公理(axiom)與定理(theorem)。\\
通常上,「對」會表示為$T$或是$\top$,而「錯」會表達為$F$或是$\bot$。在這篇文章當中,會以$\top$及$\bot$表達。\\
首先,「公理」是對該話題的預先假設,而「定理」是從假設中推論出來的,而定理會附帶證明。通常,推論會寫成$A,B\vdash C$,意味著以$A, B$為前提推論出$C$。\\
若是以$\vdash A$表達,則代表除了此定理或公理外,不需要其他前提,就可以推導出$A$。\\
\axm
\begin{equation}\label{logic:A1}
	\vdash A \to \sd{B \to A}
\end{equation}
\axm
\begin{equation}\label{logic:A2}
	\vdash \sd{A \to \sd{B \to C}} \to \sd{\sd{A \to B} \to \sd{A \to C}}
\end{equation}
\axm
\begin{equation}\label{logic:MP}\tag{MP}
	\sd{A \to B}, A \vdash B
\end{equation}
以上三個公理中,$A,B,C$是任意敘述。以上公理,可以理解成是在對$\to$做定義,只要符合以上公理形式的概念,都是可以使用的。\\
在定理的證明當中,我會以以下格式書寫:\\
\pftab{
條目&內容&前提\\
條目&內容&前提\\
條目&內容&前提\\
…&…&…\\
}
\thm
\begin{equation}\label{logic:A1MP}
	A \vdash B \to A
\end{equation}
\pf\\
\pftab{
	\rom{1}&$A$&前提\\
	\rom{2}&$A \to \sd{ B \to A}$&\ref{logic:A1}\\
	\rom{3}&$B \to A$&\rom{1}, \rom{2}, \ref{logic:MP}\\
}
\thm
\begin{equation}\label{logic:A2MP}
	A \to \sd{B \to C} \vdash \sd{A \to B} \to \sd{A \to C}\\
\end{equation}
\pf\\
\pftab{
	\rom{1}&$A \to \sd{B \to C}$&前提\\
	\rom{2}&$\sd{A \to \sd{B \to C}} \to \sd{\sd{A \to B} \to \sd{A \to C}}$&\ref{logic:A2}\\
	\rom{3}&$\sd{A \to B} \to \sd{A \to C}$&\rom{1}, \rom{2}, \ref{logic:MP}\\
}
\thm
\begin{equation}\label{logic:to_self}
	\vdash A \to A
\end{equation}
\pf\\
\pftab{
	\rom{1}&$A \to \sd{\sd{A \to A} \to A}$&\ref{logic:A1}\\
	\rom{2}&$\sd{A \to \sd{A \to A}} \to \sd{A \to A}$&\rom{1}, \ref{logic:A2MP}\\
	\rom{3}&$A \to \sd{A \to A}$&\ref{logic:A1}\\
	\rom{4}&$A \to A$&\rom{2}, \rom{3}, \ref{logic:MP}\\
}
\thm
\begin{equation}\label{logic:D1}
	\sd{A \to B}, \sd{B \to C} \vdash \sd{A \to C}
\end{equation}
\pf\\
\pftab{
	\rom{1}&$\sd{A \to B}$&前提\\
	\rom{2}&$\sd{B \to C}$&前提\\
	\rom{3}&$A \to \sd{B \to C}$&\rom{2}, \ref{logic:A1MP}\\
	\rom{4}&$\sd{A \to B} \to \sd{A \to C}$&\rom{3}, \ref{logic:A2MP}\\
	\rom{5}&$A \to C$&\rom{1}, \rom{4}, \ref{logic:MP}\\
}
\thm
\begin{equation}\label{logic:D2}
	A \to \sd{B \to C}, B \vdash \sd{A \to C}
\end{equation}
\pf\\
\pftab{
	\rom{1}&$A \to \sd{B \to C}$&前提\\
	\rom{2}&$B$&前提\\
	\rom{3}&$\sd{A \to B} \to \sd{A \to C}$&\rom{1}, \ref{logic:A2MP}\\
	\rom{4}&$A \to B$&\rom{2}, \ref{logic:A1MP}\\
	\rom{5}&$A \to C$&\rom{3}, \rom{4}, \ref{logic:MP}\\
}
\axm
\begin{equation}\label{logic:T1}
	\vdash \sd{\sd{\lnot A} \to \sd{\lnot B}} \to \sd{B \to A}
\end{equation}
這個公理是對$\lnot$的定義。\\
\thm
\begin{equation}\label{logic:T1MP}
	\sd{\lnot A} \to \sd{\lnot B} \vdash B \to A
\end{equation}
\pf\\
\pftab{
	\rom{1}&$\sd{\lnot A} \to \sd{\lnot B}$&前提\\
	\rom{2}&$\sd{\sd{\lnot A} \to \sd{\lnot B}} \to \sd{B \to A}$&\ref{logic:T1}\\
	\rom{3}&$B \to A$&\rom{1}, \rom{2}, \ref{logic:MP}\\
}
\thm
\begin{equation}\label{logic:DNe}
	\vdash \sd{\lnot\lnot A} \to A
\end{equation}
\pf\\
\pftab{
	\rom{1}&$\sd{\lnot\lnot A} \to \sd{\sd{\lnot\lnot\lnot\lnot A} \to \sd{\lnot\lnot A}}$&\ref{logic:A1}\\
	\rom{2}&$\sd{\sd{\lnot\lnot\lnot\lnot A} \to \sd{\lnot\lnot A}} \to \sd{\sd{\lnot A} \to \sd{\lnot\lnot\lnot A}}$&\ref{logic:T1}\\
	\rom{3}&$\sd{\lnot\lnot A} \to \sd{\sd{\lnot A} \to \sd{\lnot\lnot\lnot A}}$&\rom{1}, \rom{2}, \ref{logic:D1}\\
	\rom{4}&$\sd{\sd{\lnot A} \to \sd{\lnot\lnot\lnot A}} \to \sd{\sd{\lnot\lnot A} \to A}$&\ref{logic:T1}\\
	\rom{5}&$\sd{\lnot\lnot A} \to \sd{\sd{\lnot\lnot A} \to A}$&\rom{3}, \rom{4}, \ref{logic:D1}\\
	\rom{6}&$\sd{\sd{\lnot\lnot A} \to \sd{\lnot\lnot A}} \to \sd{\sd{\lnot\lnot A} \to A}$&\rom{5}, \ref{logic:A2MP}\\
	\rom{8}&$\sd{\lnot\lnot A} \to \sd{\lnot\lnot A}$&\ref{logic:to_self}\\
	\rom{9}&$\sd{\lnot\lnot A} \to A$&\rom{7}, \rom{8}, \ref{logic:MP}\\
}
\thm
\begin{equation}\label{logic:DNi}
	\vdash A \to \sd{\lnot\lnot A}
\end{equation}
\pf\\
\pftab{
	\rom{1}&$\sd{\lnot\lnot\lnot A} \to \sd{\lnot A}$&\ref{logic:DNe}\\
	\rom{3}&$ A \to \sd{\lnot\lnot A}$&\rom{1}, \ref{logic:T1MP}\\
}
\thm
\begin{equation}\label{logic:T2}
	\vdash \sd{A \to B} \to \sd{\sd{\lnot B} \to \sd{\lnot A}}
\end{equation}
\pf\\
\pftab{
	\rom{1}&$\sd{A \to B} \to \sd{\sd{\lnot\lnot A} \to \sd{A \to B}}$&\ref{logic:A1}\\
	\rom{2}&$\sd{\sd{\lnot\lnot A} \to \sd{A \to B}} \to \sd{\sd{\sd{\lnot\lnot A} \to A} \to \sd{\sd{\lnot\lnot A} \to B}}$&\ref{logic:A2}\\
	\rom{3}&$\sd{A \to B} \to \sd{\sd{\sd{\lnot\lnot A} \to A} \to \sd{\sd{\lnot\lnot A} \to B}}$&\rom{1}, \rom{2}, \ref{logic:D1}\\
	\rom{4}&$\sd{\lnot\lnot A} \to A$&\ref{logic:DNe}\\
	\rom{5}&$\sd{A \to B} \to \sd{\sd{\lnot\lnot A} \to B}$&\rom{3}, \rom{4}, \ref{logic:D2}\\
	\rom{6}&$B \to \sd{\lnot\lnot B}$&\ref{logic:DNi}\\
	\rom{7}&$\sd{\lnot\lnot A} \to \sd{B \to \sd{\lnot\lnot B}}$&\rom{6}, \ref{logic:A1MP}\\
	\rom{8}&$\sd{\sd{\lnot\lnot A} \to B} \to \sd{\sd{\lnot\lnot A} \to \sd{\lnot\lnot B}}$&\rom{7}, \ref{logic:A2MP}\\
	\rom{9}&$\sd{A \to B} \to \sd{\sd{\lnot\lnot A} \to \sd{\lnot\lnot B}}$&\rom{5}, \rom{8}, \ref{logic:D1}\\
	\rom{10}&$\sd{\sd{\lnot\lnot A} \to \sd{\lnot\lnot B}} \to \sd{\sd{\lnot B} \to \sd{\lnot A}}$&\ref{logic:T1}\\
	\rom{11}&$\sd{A \to B} \to \sd{\sd{\lnot B} \to \sd{\lnot A}}$&\rom{9}, \rom{10}, \ref{logic:D1}\\
}
\thm
\begin{equation}\label{logic:T2MP}
	A \to B \vdash \sd{\lnot B} \to \sd{\lnot A}
\end{equation}
\pf\\
\pftab{
	\rom{1}&$A \to B$&前提\\
	\rom{2}&$\sd{A \to B} \to \sd{\sd{\lnot B} \to \sd{\lnot A}}$&\ref{logic:T2}\\
	\rom{3}&$\sd{\lnot B} \to \sd{\lnot A}$&\rom{1}, \rom{2}, \ref{logic:MP}\\
}
\thm
\begin{equation}\label{logic:self_or}
	\vdash \sd{\sd{\lnot A} \to A} \to A
\end{equation}
\pf\\
\pftab{
	\rom{1}&$\sd{\sd{\lnot A} \to A} \to \sd{\sd{\lnot A} \to A}$&\ref{logic:to_self}\\
	\rom{2}&$\sd{\sd{\sd{\lnot A} \to A} \to \sd{\lnot A}} \to \sd{\sd{\sd{\lnot A} \to A} \to A}$&\rom{1}, \ref{logic:A2MP}\\
	\rom{3}&$\sd{\lnot A} \to \sd{\sd{\sd{\lnot A} \to A} \to \sd{\lnot A}}$&\ref{logic:A1}\\
	\rom{4}&$\sd{\lnot A} \to \sd{\sd{\sd{\lnot A} \to A} \to A}$&\rom{2}, \rom{3}, \ref{logic:D1}\\
	\rom{5}&$\sd{\sd{\sd{\lnot A} \to A} \to A} \to \sd{\sd{\lnot A} \to \sd{\lnot \sd{\sd{\lnot A} \to A}}}$&\ref{logic:T2}\\
	\rom{6}&$\sd{\lnot A} \to \sd{\sd{\lnot A} \to \sd{\lnot\sd{\sd{\lnot A} \to A}}}$&\rom{5}, \rom{6}, \ref{logic:D1}\\
	\rom{7}&$\sd{\sd{\lnot A} \to \sd{\lnot A}} \to \sd{\sd{\lnot A} \to \sd{\lnot\sd{\sd{\lnot A} \to A}}}$&\rom{6}, \ref{logic:A2MP}\\
	\rom{8}&$\sd{\lnot A} \to \sd{\lnot A}$&\ref{logic:to_self}\\
	\rom{9}&$\sd{\lnot A} \to \sd{\lnot\sd{\sd{\lnot A} \to A}}$&\rom{7}, \rom{8}, \ref{logic:MP}\\
	\rom{10}&$\sd{\sd{\lnot A} \to A} \to A$&\rom{9}, \ref{logic:T1MP}\\
}
\thm
\begin{equation}\label{logic:A3}
	\vdash \sd{\sd{\lnot A} \to \sd{\lnot B}} \to \sd{\sd{\sd{\lnot A} \to B} \to A}\\
\end{equation}
\pf\\
\pftab{
	\rom{1}&$\sd{\sd{\lnot A} \to A} \to A$&\ref{logic:self_or}\\
	\rom{2}&$\sd{B \to A} \to \sd{\sd{\lnot A} \to \sd{B \to A}}$&\ref{logic:A1}\\
	\rom{3}&$\sd{\sd{\lnot A} \to \sd{B \to A}} \to \sd{\sd{\sd{\lnot A} \to B} \to \sd{\sd{\lnot A} \to A}}$&\ref{logic:A2}\\
	\rom{4}&$\sd{B \to A} \to \sd{\sd{\sd{\lnot A} \to B} \to \sd{\sd{\lnot A} \to A}}$&\rom{2}, \rom{3}, \ref{logic:D1}\\
	\rom{5}&$\sd{\sd{\lnot A} \to B} \to \sd{\sd{\sd{\lnot A} \to A} \to A}$&\rom{1}, \ref{logic:A1MP}\\
	\rom{6}&$\sd{B \to A} \to \sd{\sd{\sd{\lnot A} \to B} \to \sd{\sd{\sd{\lnot A} \to A} \to A}}$&\rom{5}, \ref{logic:A1MP}\\
	\rom{7}&$\sd{\sd{\sd{\lnot A} \to B} \to \sd{\sd{\sd{\lnot A} \to A} \to A}} \to \sd{\sd{\sd{\sd{\lnot A} \to B} \to \sd{\sd{\lnot A} \to A}} \to \sd{\sd{\sd{\lnot A} \to B} \to A}}$&\ref{logic:A2}\\
	\rom{8}&$\sd{B \to A} \to \sd{\sd{\sd{\sd{\lnot A} \to B} \to \sd{\sd{\lnot A} \to A}} \to \sd{\sd{\sd{\lnot A} \to B} \to A}}$&\rom{6}, \rom{7}, \ref{logic:D1}\\
	\rom{9}&$\sd{\sd{B \to A} \to \sd{\sd{\sd{\lnot A} \to B} \to \sd{\sd{\lnot A} \to A}}} \to \sd{\sd{B \to A} \to \sd{\sd{\sd{\lnot A} \to B} \to A}}$&\rom{8}, \ref{logic:A2MP}\\
	\rom{10}&$\sd{B \to A} \to \sd{\sd{\sd{\lnot A} \to B} \to A}$&\rom{4}, \rom{9}, \ref{logic:MP}\\
}
\thm
\begin{equation}\label{logic:A4}
	A \to B, \sd{\lnot A} \to B \vdash B
\end{equation}
\pf\\
\pftab{
	\rom{1}&$A \to B$&前提\\
	\rom{2}&$\sd{\lnot A} \to B$&前提\\
	\rom{3}&$\sd{\lnot B} \to \sd{\lnot A}$&\rom{1}, \ref{logic:T2MP}\\
	\rom{4}&$\sd{\lnot B} \to B$&\rom{1}, \rom{2}, \ref{logic:D1}\\
	\rom{5}&$\sd{\sd{\lnot B} \to B} \to B$&\ref{logic:self_or}\\
	\rom{6}&$B$&\rom{4}, \rom{5}, \ref{logic:MP}\\
}
\thm
\begin{equation}\label{logic:M0}
	A, \lnot A \vdash B
\end{equation}
\pf\\
\pftab{
	\rom{1}&$A$&前提\\
	\rom{2}&$\lnot A$&前提\\
	\rom{3}&$\sd{\lnot B} \to A$&\rom{1}, \ref{logic:A1MP}\\
	\rom{4}&$\sd{\lnot B} \to \sd{\lnot A}$&\rom{2}, \ref{logic:A1MP}\\
	\rom{5}&$\sd{\lnot A} \to \sd{\lnot\lnot B}$&\rom{3}, \ref{logic:T2MP}\\
	\rom{6}&$\sd{\lnot\lnot B} \to B$&\ref{logic:DNe}\\
	\rom{7}&$\sd{\lnot A} \to B$&\rom{5}, \rom{6}, \ref{logic:D1}\\
	\rom{8}&$A \to B$&\rom{4}, \ref{logic:T1MP}\\
	\rom{9}&$B$&\rom{7}, \rom{8}, \ref{logic:A4}\\
}
\thm
\end{document}
