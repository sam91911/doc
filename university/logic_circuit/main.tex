\documentclass{article}
\usepackage{amssymb}
\usepackage{amsmath}
\usepackage{circuitikz}
\usepackage{multirow}
\usepackage[margin=2.0cm]{geometry}
\title{Logic Circuit}
\author{CHEN DING-SHAN}
\date{}
\newcommand{\sd}[1]{{\left(#1\right)}}
\newcommand{\df}[1]{\emph{def.} \textbf{#1}}
\newcommand{\thm}[1]{\emph{thm.} \textbf{#1}}
\newcommand{\anno}[1]{\emph{anno.} \emph{#1}}
\newcommand{\md}[1]{{\left[#1\right]}}
\newcommand{\prer}[1]{\emph{prer.} #1}
\newcommand{\oper}[2]{\emph{oper.} \textbf{#1} of \textbf{#2}}
\newcommand{\prop}[2]{\emph{prop.} \textbf{#1} of \textbf{#2}}
\begin{document}
\maketitle
\noindent
There are two types of basic signal in Logic Circuit: $\textbf{1}$ and $\textbf{0}$ and three types of basic circuit: and, or, not. The input quantities of and, or, not is 2, 2, 1, respectively.\\
The draw of the three types of circuit is showing below:\\
\begin{center}
\begin{tabular}{|c|c|c|}
	\hline
	and & or & not\\
	\hline
\begin{circuitikz} \draw
	(0,0) node[and port] (and) {};
\end{circuitikz}
	&
\begin{circuitikz} \draw
	(0,0) node[or port] (or) {};
\end{circuitikz}
	&
\begin{circuitikz} \draw
	(0,0) node[not port] (not) {};
\end{circuitikz}\\
	\hline
\end{tabular}
\end{center}
Below is the inout table of the three types of circuit:\\
\begin{center}
\begin{tabular}{|c|c|c|c|c|c|c|c|}
	\hline
	\multicolumn{3}{|c|}{and} & \multicolumn{3}{|c|}{or} & \multicolumn{2}{|c|}{not}\\
	\hline
	\multicolumn{2}{|c|}{input} & output & \multicolumn{2}{|c|}{input} & output & input & output\\
	\hline
	0 & 0 & 0 & 0 & 0 & 0 & \multirow{2}{*}{0} & \multirow{2}{*}{1} \\
	\cline{1-6}
	0 & 1 & 0 & 0 & 1 & 1 & & \\
	\hline
	1 & 0 & 0 & 1 & 0 & 1 & \multirow{2}{*}{1} & \multirow{2}{*}{0} \\
	\cline{1-6}
	1 & 1 & 1 & 1 & 1 & 1 & & \\
	\hline
\end{tabular}
\end{center}
To express a circuit can use several ways. Here, four ways are introduced, which are Drawing, Equation, Verilog, VHDL.\\

\end{document}
