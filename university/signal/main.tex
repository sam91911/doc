\documentclass{article}
\usepackage{amssymb}
\usepackage{amsmath}
\usepackage[margin=2.0cm]{geometry}
\title{Signal and System}
\author{CHEN DING-SHAN}
\date{}
\newcommand{\sd}[1]{{\left(#1\right)}}
\newcommand{\df}[1]{\emph{def.} \textbf{#1}}
\newcommand{\thm}[1]{\emph{thm.} #1}
\newcommand{\pf}[1]{\emph{pf.} #1}
\newcommand{\anno}[1]{\emph{anno.} \emph{#1}}
\newcommand{\md}[1]{{\left[#1\right]}}
\newcommand{\ld}[1]{{\left\{#1\right\}}}
\newcommand{\prer}[1]{\emph{prer.} #1}
\newcommand{\oper}[2]{\emph{oper.} \textbf{#1} of \textbf{#2}}
\newcommand{\prop}[2]{\emph{prop.} \textbf{#1} of \textbf{#2}}
\newcommand{\abs}[1]{{\left|#1\right|}}
\begin{document}
\maketitle
\noindent
\df{Signal}\\
Everything that carry information can be says as a signal, e.g., $R$, $C$, $R^R$, $N^N$.\\
\df{System}\\
$x$ is a System $\iff x \in A^A$ where $A$ is sets.\\
\anno{$x \in A^A$ is another presentation of "$x$ is a function mapping from $A$ to $A$". In this article, the $A^A$ presentation will be used}\\
\df{continuous time signal}\\
$x$ is continuous time signal $\iff x \in A^R$ where $A$ is a set.\\
At this article, it will be restricted to $C^R$ or $R^R$\\
\df{discrete time signal}\\
$x$ is discrete time signal $\iff x \in A^Z$ where $A$ is a set.\\
At this article, it will be restricted to $C^Z$ or $R^Z$\\
\oper{addition}{continuous time signal}\\
The addtition of the two continuous time signal is $f+g$ generated by the equation $\sd{\forall x \in R}\md{\sd{f+g}\sd{x} = f\sd{x} + g\sd{x}}$.\\
\oper{scaling}{continuous time signal}\\
The scaling of the continuous signal $f$ with factor $a \in R$ is $af$ generated by the equation $\sd{\forall x \in R}\md{\sd{af}\sd{x} = a\sd{f\sd{x}}}$.\\
\oper{addition}{discrete time signal}\\
The addtition of the two discrete time signal is $f+g$ generated by the equation $\sd{\forall x \in R}\md{\sd{f+g}\sd{x} = f\sd{x} + g\sd{x}}$.\\
\oper{scaling}{discrete time signal}\\
The scaling of the discrete signal $f$ with factor $a \in R$ is $af$ generated by the equation $\sd{\forall x \in R}\md{\sd{af}\sd{x} = a\sd{f\sd{x}}}$.\\
\oper{addition}{system}\\
The addtition of the two systems is $f+g$ generated by the equation $\sd{\forall x \in A}\md{\sd{f+g}\sd{x} = f\sd{x} + g\sd{x}}$. $A$ is the signal field.\\
\oper{scaling}{system}\\
The scaling of the system $f$ with factor $a$ is $af$ generated by the equation $\sd{\forall x \in A}\md{\sd{af}\sd{x} = a\sd{f\sd{x}}}$. $A$ is the signal field.\\
\oper{composition}{system}\\
The composition of the two system $f,g$ is $f\circ g$ generated by the equation $\sd{\forall x \in A}\md{\sd{f\circ g}\sd{x} = f\sd{g\sd{x}}}$. $A$ is the signal field.\\
\prop{linearity}{system}\\
A system $f$ is linear $\iff \sd{\forall \sd{x,y} \in A^2}\md{f\sd{x+y} = f\sd{x}+f\sd{y}}\land\sd{\forall \sd{a,x} \in R\times A}\md{f\sd{ax} = af\sd{x}}$. $A$ is the signal field.\\
\prop{time-invariant}{system}\\
A system $f$ is time-invariant \\
\begin{math}\iff \sd{\forall \sd{x,y} \in A^2}\left[\sd{\exists t_0 \in B}\md{\sd{\forall t \in B}\md{x\sd{t} = y\sd{t+t_0}}} \right.\\ \left. \to \sd{\exists t_0 \in B}\md{\sd{\forall t \in B}\md{x\sd{t} = y\sd{t+t_0}}\land\sd{\forall t \in B}\md{\sd{f\sd{x}}\sd{t} = \sd{f\sd{y}}\sd{t+t_0}}}\right]\end{math}.
$A$ is the signal field, and $B$ is the domain of the signals.\\
Plainly, $f$ is time-invariant if and only if for any signal pair that $y$ has a time sift of $x$, the system output $f\sd{y}$ and $f\sd{x}$ will remain the same time sift.\\
\prop{LTI (linear time-invariant)}{system}\\
A system $f$ is LTI $\iff$ $f$ is linear and $f$ is time-invariant\\
At this article, it will be restricted to LTI system.\\
The linear algebra tells us that if a function is linear, it might have eigen vectors. Let's find out the eigen vectors of LTI system.\\

\end{document}
