\documentclass{article}
\usepackage{amssymb}
\usepackage{amsmath}
\usepackage[margin=2.0cm]{geometry}
\title{Signal and System}
\author{CHEN DING-SHAN}
\newcommand{\sd}[1]{\left(#1\right)}
\newcommand{\df}[1]{\emph{def.} \textbf{#1}}
\newcommand{\thm}[1]{\emph{thm.} \textbf{#1}}
\newcommand{\anno}[1]{\emph{anno.} \emph{#1}}
\newcommand{\md}[1]{\left[#1\right]}
\begin{document}
\df{Signal}\\
$x$ is a Signal $\iff x \in A^B$ where $A, B$ is sets, or it can be represented by $x$ is a function, $f:B -> A$\\
\anno{the element of $A^B$ is functions $f:B -> A$}\\
\df{System}\\
$x$ is a System $\iff x \in \sd{A^B}^\sd{A^B}$ where $A, B$ is sets\\
\df{continuous time signal}\\
$x$ is ... $\iff x \in A^R$ where $A$ is a set\\
\df{discrete time signal}\\
$x$ is ... $\iff x \in A^Z$ where $A$ is a set\\
at this scope, the codomain is typically $R$, e.g., a continuous time signal is a element in $R^R$\\
\df{addition of signal}\\
The domain $A$ and codomain $B$ of two signals, $f,g$, is the same, and $B$ has addition operation.\\
$\implies$ The addition of the two signal is $f+g$ generated by the equaltion $\sd{\forall x}\md{\sd{x \in A} \to \sd{\sd{f+g}\sd{x} = f\sd{x} + g\sd{x}}$.\\
\end{document}
